\section{Introduction.}

\noindent The Vera C. Rubin Observatory\footnote{About the Vera C. Rubin Observatory: \url{https://www.lsst.org/about}}, scheduled to start operations on 2023, hosts the Simonyi Survey Telescope. With a primary mirror of 8.4 m in diameter, the telescope will conduct the \textbf{Legacy Survey of Space and Time} by observing the southern night sky for 10 years. The survey will focus on monitoring the transient phenomena of the deep sky and the Solar System, probing dark energy and dark matter, mapping the Milky Way, and more.\\

\noindent The Rubin Observatory is located in Cerro Pachón, about 100 km from La Serena, Chile, and it is now in the final stages of construction and setup. One of the most important parts of the telescope is its camera: the \textbf{LSSTCam}, a mosaic of 189 CCDs with a 3.2 gigapixel resolution designed and built by the SLAC National Accelerator Laboratory at Stanford University.\\

\noindent In order to process the massive amount of data that the LSST will generate ($\sim 20$ TB per night) the Rubin Observatory staff have developed the \textbf{LSST Science Pipelines} (unofficialy called `The Stack')\footnote{About the Stack: \url{https://pipelines.lsst.io/}}. In this internship we worked using the Stack to process several PTC data from the LSSTCam's CCDs. Our main objective for this project was to learn how to handle PTC data and try several possible setup parameters for the linearization algorithm developed in the Stack. \\

\noindent This document is a report of the work we developed during the internship with funding from the \textbf{LSSTC Enabling Science Award 2021-51}, and presents our main results. In sections \ref{PTC1} and \ref{Diodes} we show our exploration of the first sets of PTC data for detector 22, and inspect how it deviates from polynomial and exponential fits. In section \ref{RSP} we present a tutorial on how to connect to the Rubin Science Platform and access the notebooks and other project files. Then in section \ref{linearizer} we present the approach taken to work with the linearizer algorithm and the results obtained by varying the parameters involved in the different fits that can be used.\\

\noindent This work was developed as one of the research projects for the \textbf{RECA Internship Program 2022}\footnote{RECA Internships: \url{https://www.astroreca.org/en/internship}} funded by the LSST Corporation and under the advice and guidance of \textbf{Dr. Craig S. Lage} and \textbf{Dr. Andrés Plazas-Malagón}.\\